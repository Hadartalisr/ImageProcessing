\documentclass[a4paper, 12pt]{article}

\usepackage[utf8]{inputenc}
\usepackage[T1]{fontenc}
\usepackage[english]{babel}
\usepackage{amsmath, amssymb}
\usepackage{graphicx}
\usepackage{listings}
\usepackage[margin=0.75in, top=0.75in, bottom=0.75in]{geometry}  % Reduce all margins
\usepackage{color}
\usepackage{float}
\usepackage{hyperref}

% Define colors
\definecolor{myblue}{RGB}{0, 0, 128}
\setlength{\parindent}{0pt}



% Define code listing settings
\lstset{
    basicstyle=\ttfamily\small,
    breaklines=true,
    numbers=left,
    numberstyle=\tiny,
    frame=single, 
    language=Python,
}

\title{Image Processing - Excercise 3}
\author{Hadar Tal, hadar.tal, 207992728}

\begin{document}

\maketitle

\section*{Introduction}

The objective of the exercise is to develop a comprehensive image registration and blending algorithm 
capable of aligning and seamlessly blending two images of varying resolutions. 
This task is tackled through a combination of advanced techniques. 
Firstly, a multiscale image pyramid is constructed to represent the images at different resolutions, 
enabling feature detection and matching across multiple scales. 
Harris corner detection is employed to identify key features in the images, 
and MOPS descriptors are generated to robustly describe these features. 
Subsequently, feature matching is performed to establish correspondences between the images, 
facilitating the estimation of geometric transformations. 
To handle outliers and ensure robustness in the presence of noise, RANSAC (Random Sample Consensus) 
is employed for accurate transformation parameter estimation. 
The estimated transformation is then applied to one of the images to align it with the other image effectively. 
Finally, image blending is executed using a mask to seamlessly integrate the transformed image with the reference image, 
ensuring a visually appealing composite result without artifacts at the boundaries. 
Overall, this exercise integrates concepts from image processing, computer vision, 
and geometric transformations to achieve the goal of accurate image registration and blending.






\section*{Algorithm}

I have implemented the following Python functions:

\section*{Implementation Details}

I have implemented the following Python functions:

\section*{Visual Results}

I have implemented the following Python functions:

\section*{Conclusion}

I have implemented the following Python functions:

\end{document}